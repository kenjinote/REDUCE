\documentclass[12pt]{article}

\newcommand{\ttindex}[1]{{\renewcommand{\_}{\protect\underscore}%
                          \index{#1@{\tt #1}}}}

\title{MODSR: Modular solve and roots}

\author{
Herbert Melenk \\
Konrad--Zuse--Zentrum f\"ur Informationstechnik Berlin \\
Takustra\"se 7 \\
D--14195 Berlin--Dahlem, Germany \\[0.05in]
e--mail: melenk@zib.de
}

\begin{document}
\maketitle

This package supports the SOLVE and ROOTS operators for modular polynomials
and modular polynomial systems.  The moduli need not be primes.  {\tt
M\_SOLVE} requires a modulus to be set.  {\tt M\_ROOTS} takes the
modulus as a second argument. For example:

\begin{verbatim}
on modular; setmod 8;
m_solve(2x=4);            ->  {{X=2},{X=6}}
m_solve({x^2-y^3=3});
    ->  {{X=0,Y=5}, {X=2,Y=1}, {X=4,Y=5}, {X=6,Y=1}}
m_solve({x=2,x^2-y^3=3}); ->  {{X=2,Y=1}}
off modular;
m_roots(x^2-1,8);         ->  {1,3,5,7}
m_roots(x^3-x,7);         ->  {0,1,6}
\end{verbatim}

\end{document}
