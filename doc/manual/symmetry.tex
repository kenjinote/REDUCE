
\index{Gatermann, Karin}\index{People!Gatermann, Karin}

\iffalse
This short note describes a package of \REDUCE{} procedures
that compute symmetry-adapted bases and block diagonal forms
of matrices which have the symmetry of a group.
The \package{SYMMETRY} package is the implementation
of the theory of linear representations
for small finite groups such as the dihedral groups.
\fi

\subsection{Introduction}

The exploitation of symmetry is a very important principle in mathematics,
physics and engineering sciences.
The aim of the \package{SYMMETRY} package is to give an easy access to the
underlying theory of linear representations for small groups. For
example the
dihedral groups $D_3,D_4,D_5,D_6$ are included.
For an introduction to the theory see {\sc Serre} \cite{Serre:77} or
{\sc Stiefel} and {\sc F\"assler} \cite{StiefelFaessler:79}.
For a given orthogonal (or unitarian) linear representation
\[
\vartheta : G\longrightarrow GL(K^n), \qquad K=R,C.
\]
the character $\psi\rightarrow K$, the
canonical decomposition or the bases of the isotypic
components are computed. A matrix $A$ having the symmetry of a linear
representation,e.g.
\[
\vartheta_t A = A \vartheta_t \quad \forall \, t\in G,
\]
is transformed to block diagonal form by a coordinate
transformation.
The dependence of the algorithm on the
field of real or complex numbers is controled by the switch \sw{complex}.
An example for this is given in the testfile \emph{symmetry.tst}.

As the algorithm needs information concerning the irreducible representations
this information is stored for some groups (see the operators in Section 3).
It is assumed that only orthogonal (unitar) representations are given.

The package is loaded by

\texttt{load symmetry;}

\subsection{Operators for linear representations}
\hypertarget{operator:CANONICALDECOMPOSITION}{}
\hypertarget{operator:CHARACTER}{}
\hypertarget{operator:SYMMETRYBASIS}{}
\hypertarget{operator:SYMMETRYBASISPART}{}
\hypertarget{operator:ALLSYMMETRYBASES}{}
\hypertarget{operator:DIAGONALIZE}{}

First the data structure for a linear representation has to be explained.
\emph{representation} is a list consisting of the group identifier and
equations which assign matrices to the generators of the group.

\textbf{Example:}
\begin{verbatim}
   rr:=mat((0,1,0,0),
           (0,0,1,0),
           (0,0,0,1),
           (1,0,0,0));

   sp:=mat((0,1,0,0),
           (1,0,0,0),
           (0,0,0,1),
           (0,0,1,0));

   representation:={D4,rD4=rr,sD4=sp};
\end{verbatim}

For orthogonal (unitarian) representations the following operators
are available.

\texttt{canonicaldecomposition(representation);}\ttindextype[SYMMETRY]{canonicaldecomposition}{operator}

returns an equation giving the canonical decomposition of the linear
representation.

\texttt{character(representation);}\ttindextype[SYMMETRY]{character}{operator}

computes the character of the linear representation. The result is a list
of the group identifier and of lists consisting of a
list of group elements in one equivalence class and a real or complex number.

\texttt{symmetrybasis(representation,nr);}\ttindextype[SYMMETRY]{symmetrybasis}{operator}

computes the basis of the isotypic component corresponding to the irreducible
representation of type nr. If the nr-th irreducible representation is
multidimensional, the basis is symmetry adapted. The output is a matrix.

\texttt{symmetrybasispart(representation,nr);}\ttindextype[SYMMETRY]{symmetrybasispart}{operator}

is similar as \texttt{symmetrybasis}, but for multidimensional
irreducible representations only the first part of the
symmetry adapted basis is computed.

\texttt{allsymmetrybases(representation);}\ttindextype[SYMMETRY]{allsymmetrybases}{operator}

is similar as \texttt{symmetrybasis} and \texttt{symmetrybasispart},
but the bases of all
isotypic components are computed and thus a
complete coordinate transformation is returned.

\texttt{diagonalize(matrix,representation);}\ttindextype[SYMMETRY]{diagonalize}{operator}

returns the block diagonal form of matrix which has the symmetry
of the given linear representation. Otherwise an error message occurs.

\texttt{on complex;}

Of course the property of irreducibility depends on the field $K$ of
real or complex numbers. This is why the algorithm depends on $K$.
The type of computation is set by the switch \sw{complex}.

\subsection{Display Operators}
\hypertarget{operator:AVAILABLEGROUPS}{}
\hypertarget{operator:PRINTGROUP}{}
\hypertarget{operator:GENERATORS}{}
\hypertarget{operator:CHARACTERTABLE}{}
\hypertarget{operator:CHARACTERN}{}
\hypertarget{operator:IRREDUCIBLEREPTABLE}{}
\hypertarget{operator:IRREDUCIBLEREPNR}{}

In this section the operators are described which give access to the
stored information for a group.
First the operators for the abstract groups are given.
Then it is described how to get the irreducible representations
for a group.

\texttt{availablegroups();}\ttindextype[SYMMETRY]{availablegroups}{operator}

returns the list of all groups for which the information such as
irreducible representations is stored. In the following \texttt{group}
is always one of these group identifiers.

\texttt{printgroup(group);}\ttindextype[SYMMETRY]{printgroup}{operator}

returns the list of all group elements;

\texttt{generators(group);}\ttindextype[SYMMETRY]{generators}{operator}

returns a list of group elements which generates the group. For the
definition of a linear representation matrices for these generators
have to be defined.

\texttt{charactertable(group);}\ttindextype[SYMMETRY]{charactertabl}{operator}

returns a list of the characters corresponding to the irreducible
representations of this group.

\texttt{charactern(group,nr);}\ttindextype[SYMMETRY]{charactern}{operator}

returns the character corresponding to the nr-th irreducible representation
of this group as a list (see also \texttt{character}).

\texttt{irreduciblereptable(group);}\ttindextype[SYMMETRY]{irreduciblereptable}{operator}

returns the list of irreducible representations of the group.

\texttt{irreduciblerepnr(group,nr);}\ttindextype[SYMMETRY]{irreduciblerepnr}{operator}

returns an irreducible representation of the group. The output
is a list of the group identifier and equations
assigning the representation matrices to group elements.

\subsection{Storing a new group}
\hypertarget{operator:SETGENERATORS}{}
\hypertarget{operator:SETELEMENTS}{}
\hypertarget{operator:SETGROUPTABLE}{}
\hypertarget{operator:RSETREPRESENTATION}{}
\hypertarget{operator:CSETREPRESENTATION}{}
\hypertarget{operator:SETAVAILABLE}{}
\hypertarget{operator:STOREGROUP}{}
\hypertarget{operator:LOADGROUPS}{}

If the user wants to do computations for a group for which
information is not predefined,
the package \package{SYMMETRY} offers the possibility to supply information
for this group.

For this the following data structures are used.

\textbf{elemlist} = list of identifiers.

\textbf{relationlist}  = list of equations with identifiers and
operators $@$ and $**$.

\textbf{grouptable} = matrix with the (1,1)-entry grouptable.

\textbf{filename} = "myfilename.new".

\vspace{2cm}
The following operators have to be used in this order.

\texttt{setgenerators(group,elemlist,relationlist);}\ttindextype[SYMMETRY]{setgenerators}{operator}

\textbf{Example:}
\begin{verbatim}
   setgenerators(K4,{s1K4,s2K4},
     {s1K4^2=id,s2K4^2=id,s1K4@s2K4=s2K4@s1K4});
\end{verbatim}

\textbf{setelements(group,relationlist);}\ttindextype[SYMMETRY]{setelements}{operator}

The group elements except the neutral element
are given as product of the defined
generators. The neutral element is always called \texttt{id}.

\textbf{Example:}
\begin{verbatim}
   setelements(K4,
        {s1K4=s1K4,s2K4=s2K4,rK4=s1K4@s2K4});
\end{verbatim}

\textbf{setgrouptable(group,grouptable);}\ttindextype[SYMMETRY]{setgrouptable}{operator}

installs the group table.

\textbf{Example:}
\begin{verbatim}
   tab:=
    mat((grouptable,     id,    s1K4, s2K4, rK4),
        (id        ,     id,    s1K4, s2K4, rK4),
        (s1K4      ,    s1K4,     id,  rK4,s2K4),
        (s2K4      ,    s2K4,    rK4,   id,s1K4),
        (rK4       ,     rK4,   s2K4, s1K4,  id));

   setgrouptable(K4,tab);
\end{verbatim}

\textbf{Rsetrepresentation(representation,type);}\ttindextype[SYMMETRY]{Rsetrepresentation}{operator}

is used to define the real irreducible representations of the group.
The variable \texttt{type} is either \emph{realtype} or \emph{complextype}
which indicates the type of the real irreducible representation.

\textbf{Example:}
\begin{verbatim}
   eins:=mat((1));
   mineins:=mat((-1));
   rep3:={K4,s1K4=eins,s2K4=mineins};
   Rsetrepresentation(rep3,realtype);
\end{verbatim}

\textbf{Csetrepresentation(representation);}\ttindextype[SYMMETRY]{Csetrepresentation}{operator}

This defines the complex irreducible representations.

\textbf{setavailable(group);}\ttindextype[SYMMETRY]{setavailable}{operator}

terminates the installation of the group203. It checks some properties of the
irreducible representations and makes the group available for the
operators in Sections 2 and 3.

\textbf{storegroup(group,filename);}\ttindextype[SYMMETRY]{storegroup}{operator}

writes the information concerning the group to the file with name
\emph{filename}.

\textbf{loadgroups(filename);}\ttindextype[SYMMETRY]{loadgroups}{operator}

loads a user defined group from the file \emph{filename} into
the system.
