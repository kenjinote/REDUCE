\chapter{Series expansion}

Expanding an algebraic expression into a series can be done by standard \REDUCE operators, namely \f{df}, \f{sub}, and possibly \f{limit}.
Nevertheless, there are many cases where this straightforward method fails.
\REDUCE offers two different operators for this purpose:
\begin{description}
\item[\f{taylor}] computes a truncated power series.
\item[\f{ps}] computes extendible power series.
  \item[\f{fps}] computes formal power series.
\end{description}

\section{Taylor expansion}
This package carries out the Taylor expansion of an expression in one
or more variables and efficient manipulation of the resulting Taylor
series. Capabilities include basic operations (addition, subtraction,
multiplication and division) and also application of certain algebraic
and transcendental functions.\footnote{This code was written by Rainer Schöpf.}


\index{Sch\"opf, Rainer}
\index{People!Sch\"opf, Rainer}

The most important operator is \texttt{taylor}.\ttindextype{taylor}{operator}
It is used as follows:
\hypertarget{operator:TAYLOR}{}
\begin{syntax}
  \texttt{taylor(}\meta{exp:algebraic}\texttt{,}
         \meta{var:kernel}\texttt{,}\meta{var0:algebraic}\texttt{,}\meta{order:integer}[,\ldots]\texttt{)}
         \,:\,algebraic.
\end{syntax}
where \f{exp} is the expression to be expanded. It can be any \REDUCE{}
object, even an expression containing other Taylor kernels. \f{var} is
the kernel with respect to which \f{exp} is to be expanded. \f{var0}
denotes the point about which and \f{order} the order up to which
expansion is to take place. If more than one \f{(var, var0, order)} triple
is specified \f{taylor} will expand its first argument independently
with respect to each variable in turn. For example,
\begin{verbatim}
  taylor(e^(x^2+y^2),x,0,2,y,0,2);
\end{verbatim}
will calculate the Taylor expansion up to order $X^{2}*Y^{2}$:
\begin{verbatim}
       2    2    2  2      3  3
  1 + y  + x  + y *x  + O(x ,y )
\end{verbatim}
Note that once the expansion has been done it is not possible to
calculate higher orders.
Instead of a kernel, \f{var} may also
be a list of kernels. In this case expansion will take place in a way
so that the \emph{sum} of the degrees of the kernels does not exceed
\f{order}.
If \f{var0} evaluates to the special identifier \f{infinity}, expansion is
done in a series in 1/var instead of \f{var}.

The expansion is performed variable per variable, i.e.\ in the example
above by first expanding $\exp(x^{2}+y^{2})$ with respect to $x$ and
then expanding every coefficient with respect to $y$.

\ttindextype{implicit\_taylor}{operator}
\hypertarget{operator:IMPLICIT_TAYLOR}{}
There are two
extra operators to compute the Taylor expansions of implicit and
inverse functions:
\begin{syntax}
  \texttt{implicit\_taylor(}\meta{f:algebraic}\texttt{,}\\
  \hspace*{1.4in}\meta{var:kernel}\texttt{,}\meta{depvar:kernel}\texttt{,}\\
  \hspace*{1.4in}\meta{var0:algebraic}\texttt{,}\meta{depvar0:algebraic}\texttt{,}\\
  \hspace*{1.4in}\meta{order:integer}\texttt{)}
         \,:\,algebraic.
\end{syntax}
takes a function f depending on two variables var and depvar and
computes the Taylor series of the implicit function depvar(var)
given by the equation f(var,depvar) = 0, around the point var0.  
(Violation of the necessary condition f(var0,depvar0)=0 causes an error.)
For example,
\begin{verbatim}
  implicit_taylor(x^2 + y^2 - 1,x,y,0,1,5);
\end{verbatim}
gives the output
\begin{verbatim}
       1   2    1   4      6
  1 - ---*x  - ---*x  + O(x )
       2        8
\end{verbatim}

\hypertarget{operator:INVERSE_TAYLOR}{}
The operator
\begin{syntax}
  \texttt{inverse\_taylor(}\meta{f:algebraic}\texttt{,}\\
  \hspace*{1.4in}\meta{var:kernel}\texttt{,}\meta{depvar:kernel}\texttt{,}\\
  \hspace*{1.4in}\meta{var0:algebraic}\texttt{,}\meta{order:integer}\texttt{)}\\
         \,:\,algebraic.
\end{syntax}
takes a function f depending on var and computes the Taylor series of
the inverse of f with respect to var0. For example,
\begin{verbatim}
  inverse_taylor(exp(x)-1,x,y,0,8);
\end{verbatim}
yields
\begin{verbatim}
       1   2    1   3    1   4    1   5                  9
  y - ---*y  + ---*y  - ---*y  + ---*y  + (3 terms) + O(y )
       2        3        4        5
\end{verbatim}


\ttindextype{taylorprintterms}{variable}\hypertarget{reserved:TAYLORPRINTTERMS}{}
When a Taylor kernel is printed, only a certain number of (non-zero)
coefficients are shown. If there are more, an expression of the form
\f{($n$ terms)} is printed to indicate how many non-zero
terms have been suppressed. The number of terms printed is given by
the value of the shared algebraic variable \var{taylorprintterms}.
Allowed values are integers and the special identifier \f{ALL}. The
latter setting specifies that all terms are to be printed. The default
setting is $5$.

\ttindextype{part}{operator!use on Taylor kernel}%
The \f{part} operator can be used to extract subexpressions of a
Taylor expansion in the usual way. All terms can be accessed,
irregardless of the value of the variable \var{taylorprintterms}.


\ttindexswitch[TAYLOR]{taylorkeeporiginal}
If the switch \hyperlink{switch:TAYLORKEEPORIGINAL}{\f{taylorkeeporiginal}}
is set to \f{on} the
original expression exp is kept for later reference.
It can be recovered by means of the operator

\hypertarget{operator:TAYLORORIGINAL}{}
\hspace*{2em} \texttt{taylororiginal}(exp:{\em exprn}):{\em exprn}

An error is signalled if exp is not a Taylor kernel or if the original
expression was not kept, i.e.\ if \sw{taylorkeeporiginal} was
\f{off} during expansion.  The template of a Taylor kernel, i.e.\
the list of all variables with respect to which expansion took place
together with expansion point and order can be extracted using
\ttindextype{taylortemplate}{operator}.

\hypertarget{operator:TAYLORTEMPLATE}{}
\hspace*{2em} \texttt{taylortemplate}(exp:{\em exprn}):{\em list}

This returns a list of lists with the three elements (var,var0,order).
As with \f{taylororiginal},
an error is signalled if exp is not a Taylor kernel.

The operator
\hypertarget{operator:TAYLORTOSTANDARD}{}\\
\hspace*{2em} \texttt{taylortostandard}(exp:{\em exprn}):{\em exprn}

converts all Taylor kernels in exp into standard form and
\ttindextype{taylortostandard}{operator} resimplifies the result.

The boolean operator
\hypertarget{operator:TAYLORSERIESP}{}\\
\hspace*{2em} \texttt{taylorseriesp}(exp:{\em exprn}):{\em boolean}

may be used to determine if exp is a Taylor kernel.
\ttindextype{taylorseriesp}{operator} (Note that this operator is subject to the same
restrictions as, e.g., \f{ordp} or \f{numberp}, i.e.\ it may only be used in
boolean expressions in \f{if} or \f{let} statements. 

Finally there is

\hypertarget{operator:TAYLORCOMBINE}{}
\hspace*{2em} \texttt{taylorcombine}(exp:{\em exprn}):{\em exprn}

which tries to combine all Taylor kernels found in exp into one.
\ttindextype{taylorcombine}{operator}
Operations currently possible are:
\index{Taylor series!arithmetic}
\begin{itemize}
  \item Addition, subtraction, multiplication, and division.
  \item Roots, exponentials, and logarithms.
  \item Trigonometric and hyperbolic functions and their inverses.
\end{itemize}
Application of unary operators like \f{log} and \f{atan} will
nearly always succeed. For binary operations their arguments have to be
Taylor kernels with the same template. This means that the expansion
variable and the expansion point must match. Expansion order is not so
important, different order usually means that one of them is truncated
before doing the operation.

\ttindexswitch[TAYLOR]{taylorkeeporiginal} \ttindextype{taylorcombine}{operator}
If \hyperlink{switch:TAYLORKEEPORIGINAL}{\f{taylorkeeporiginal}} is set to \f{on} and if all Taylor
kernels in \f{exp} have their original expressions kept
\hyperlink{operator:TAYLORCOMBINE}{\f{taylorcombine}} will also combine these and store the result
as the original expression of the resulting Taylor kernel.
\ttindexswitch[TAYLOR]{taylorautoexpand}
There is also the switch \hyperlink{switch:TAYLORAUTOEXPAND}{\f{taylorautoexpand}} (see below).

There are a few restrictions to avoid mathematically undefined
expressions: it is not possible to take the logarithm of a Taylor
kernel which has no terms (i.e. is zero), or to divide by such a
beast.  There are some provisions made to detect singularities during
expansion: poles that arise because the denominator has zeros at the
expansion point are detected and properly treated, i.e.\ the Taylor
kernel will start with a negative power.  (This is accomplished by
expanding numerator and denominator separately and combining the
results.)  Essential singularities of the known functions (see above)
are handled correctly.

\index{Taylor series!differentiation}
Differentiation of a Taylor expression is possible.  If you
differentiate with respect to one of the Taylor variables the order
will decrease by one.

\index{Taylor series!substitution}
Substitution is a bit restricted: Taylor variables can only be replaced
by other kernels.  There is one exception to this rule: you can always
substitute a Taylor variable by an expression that evaluates to a
constant.  Note that \REDUCE{} will not always be able to determine
that an expression is constant.

\index{Taylor series!integration}
Only simple taylor kernels can be integrated. More complicated
expressions that contain Taylor kernels as parts of themselves are
automatically converted into a standard representation by means of the
\hyperlink{operator:TAYLORTOSTANDARD}{\f{taylortostandard}} operator. 
In this case a suitable warning is printed.

\index{Taylor series!reversion} It is possible to revert a Taylor
series of a function $f$, i.e., to compute the first terms of the
expansion of the inverse of $f$ from the expansion of $f$. This is
done by the operator

\hypertarget{operator:TAYLORREVERT}{}
\ttindextype{taylorrevert}{operator}
\hspace*{2em} \texttt{taylorrevert}(exp:{\em exprn},oldvar:{\em kernel},
                                 NEWVAR:{\em kernel}):{\em exprn}

EXP must evaluate to a Taylor kernel with OLDVAR being one of its
expansion variables. Example:
\begin{verbatim}
  taylor (u - u**2, u, 0, 5)$
  taylorrevert (ws, u, x);
\end{verbatim}
gives
\begin{verbatim}
       2      3      4       5      6
  x + x  + 2*x  + 5*x  + 14*x  + O(x )
\end{verbatim}

This package introduces a number of new switches:
\begin{description}

\ttindexswitch[TAYLOR]{taylorautocombine}
\item[\sw{taylorautocombine}] \hypertarget{switch:TAYLORAUTOCOMBINE}{}causes
    Taylor expressions to be automatically combined during the simplification
    process.  This is equivalent to applying \f{taylorcombine} to
    every expression that contains Taylor kernels.
    Default is \f{on}.

\ttindexswitch[TAYLOR]{taylorautoexpand}
\item[\sw{taylorautoexpand}] \hypertarget{switch:TAYLORAUTOEXPAND}{} makes Taylor expressions ``contagious''
    in the sense that \f{taylorcombine} tries to Taylor expand
    all non-Taylor subexpressions and to combine the result with the
    rest. Default is \f{off}.

\ttindexswitch[TAYLOR]{taylorkeeporiginal}\hypertarget{switch:TAYLORKEEPORIGINAL}{}
\item[\sw{taylorkeeporiginal}] forces the
    package to keep the original expression, i.e.\ the expression
    that was Taylor expanded.  All operations performed on the
    Taylor kernels are also applied to this expression  which can
    be recovered using the operator \f{taylororiginal}.
    Default is \f{off}.

\ttindexswitch[TAYLOR]{taylorprintorder}\hypertarget{switch:TAYLORPRINTORDER}{}
\item[\sw{taylorprintorder}] causes the
    remainder to be printed in big-$O$ notation.  Otherwise, three
    dots are printed. Default is \f{on}.

\ttindexswitch[TAYLOR]{verboseload}
\item[\sw{verboseload}] will cause
    \REDUCE{} to print some information when the Taylor package is
    loaded.  This switch is already present in \textsf{PSL} systems.
    Default is \f{off}.

\end{description}
\index{\textsc{TAYLOR} package!Defaults}

\subsection{Caveats}
\index{\textsc{TAYLOR} package!Caveats}
 \index{Caveats!\textsc{TAYLOR} package}

\f{taylor} should always detect non-analytical expressions in
its first argument. As an example, consider the function $xy/(x+y)$
that is not analytical in the neighborhood of $(x,y) = (0,0)$: Trying
to calculate
\begin{verbatim}
  taylor(x*y/(x+y),x,0,2,y,0,2);
\end{verbatim}
causes an error
\begin{verbatim}
***** Not a unit in argument to QUOTTAYLOR
\end{verbatim}
Note that it is not generally possible to apply the standard \REDUCE{}
operators to a Taylor kernel. For example, \f{coeff}
or \f{coeffn} cannot be used. Instead, the expression at hand has
to be converted to standard form first using the \f{taylortostandard}
operator.

\subsection{Warning messages}
\index{Warnings!TAYLOR package}
\begin{description}

\item[\msg{*** Cannot expand further... truncation done}]\mbox{}\\
    You will get this warning if you try to expand a Taylor kernel to
    a higher order.

\item[\msg{*** Converting Taylor kernels to standard representation}]\mbox{}\\
    This warning appears if you try to integrate an expression
    containing Taylor kernels.

\end{description}

\subsection{Error messages}
\index{Errors!TAYLOR package}
\begin{description}

\item[\msg{***** Branch point detected in ...}]\mbox{}\\
    This occurs if you take a rational power of a Taylor kernel
    and raising the lowest order term of the kernel to this
    power yields a non analytical term (i.e.\ a fractional power).

\item[\msg{***** Cannot replace part ... in Taylor kernel}]\mbox{}\\
\ttindextype{part}{operator!error when using on Taylor kernel}%
    The \f{part} operator can only be used to either replace the
    template of a Taylor kernel (part 2) or the original expression
    that is kept for reference (part 3).    

\item[\msg{***** Computation loops (recursive definition?): ...}]\mbox{}\\
    Most probably the expression to be expanded contains an operator
    whose derivative involves the operator itself.

\item[\msg{***** Error during expansion (possible singularity)}]\mbox{}\\
    The expression you are trying to expand caused an error.
    As far as I know this can only happen if it contains a function
    with a pole or an essential singularity at the expansion point.
    (But one can never be sure.)

\item[\msg{***** Essential singularity in ...}]\mbox{}\\
    An essential singularity was detected while applying a
    special function to a Taylor kernel.

\item[\msg{***** Expansion point lies on branch cut in ...}]\mbox{}\\
    The only functions with branch cuts this package knows of are
    (natural) logarithm, inverse circular and hyperbolic tangent and
    cotangent.  The branch cut of the logarithm is assumed to lie on
    the negative real axis.  Those of the arc tangent and arc
    cotangent functions are chosen to be compatible with this: both
    have essential singularities at the points $\pm i$.  The branch
    cut of arc tangent is the straight line along the imaginary axis
    connecting $+1$ to $-1$ going through $\infty$ whereas that of arc
    cotangent goes through the origin.  Consequently, the branch cut
    of the inverse hyperbolic tangent resp.\ cotangent lies on the
    real axis and goes from $-1$ to $+1$, that of the latter across
    $0$, the other across $\infty$.

    The error message can currently only appear when you try to
    calculate the inverse tangent or cotangent of a Taylor
    kernel that starts with a negative degree.
    The case of a logarithm of a Taylor kernel whose constant term
    is a negative real number is not caught since it is
    difficult to detect this in general.

\item[\msg{***** Input expression non-zero at given point}]\mbox{}\\
    Violation of the necessary condition f(var0,depvar0)=0 for the arguments of
    \f{implicit\_taylor}.

\item[\msg{***** Invalid substitution in Taylor kernel: ...}]\mbox{}\\
    You tried to substitute a variable that is already present in the
    Taylor kernel or on which one of the Taylor variables depend.

\item[\msg{***** Not a unit in ...}]\mbox{}\\
    This will happen if you try to divide by or take the logarithm of
    a Taylor series whose constant term vanishes.

\item[\msg{***** Not implemented yet (...)}]\mbox{}\\
    Sorry, but I haven't had the time to implement this feature.
    Tell me if you really need it, maybe I have already an improved
    version of the package.

\item[\msg{***** Reversion of Taylor series not possible: ...}]\mbox{}\\
\ttindextype{taylorrevert}{operator}
    You tried to call the \f{taylorrevert} operator with
    inappropriate arguments. The second half of this error message
    tells you why this operation is not possible.

\item[\msg{***** Taylor kernel doesn't have an original part}]\mbox{}\\
\ttindextype{taylororiginal}{operator} \ttindexswitch[TAYLOR]{taylorkeeporiginal}
    The Taylor kernel upon which you try to use \f{taylororiginal}
    was created with the switch \sw{taylorkeeporiginal}
    set to \f{off}
    and does therefore not keep the original expression.

\item[\msg{***** Wrong number of arguments to TAYLOR}]\mbox{}\\
    You try to use the operator \f{taylor} with a wrong number of
    arguments.

\item[\msg{***** Zero divisor in TAYLOREXPAND}]\mbox{}\\
    A zero divisor was found while an expression was being expanded.
    This should not normally occur.

\item[\msg{***** Zero divisor in Taylor substitution}]\mbox{}\\
    That's exactly what the message says.  As an example consider the
    case of a Taylor kernel containing the term \f{1/x} and you try
    to substitute \f{x} by \f{0}.

\item[\msg{***** ... invalid as kernel}]\mbox{}\\
    You tried to expand with respect to an expression that is not a
    kernel.

\item[\msg{***** ... invalid as order of Taylor expansion}]\mbox{}\\
    The order parameter you gave to \f{taylor} is not an integer.

\item[\msg{***** ... invalid as Taylor kernel}]\mbox{}\\
\ttindextype{taylororiginal}{operator} \ttindextype{taylortemplate}{operator}
    You tried to apply \f{taylororiginal} or \f{taylortemplate}
    to an expression that is not a Taylor kernel.

\item[\msg{***** ... invalid as Taylor Template element}]\mbox{}\\
    You tried to substitute the \f{taylortemplate} part of a Taylor
    kernel with a list a incorrect form. For the correct form see the
    description of the \f{taylortemplate} operator.

\item[\msg{***** ... invalid as Taylor variable}]\mbox{}\\
    You tried to substitute a Taylor variable by an expression that is
    not a kernel.

\item[\msg{***** ... invalid as value of TaylorPrintTerms}]\mbox{}\\
\ttindextype{taylorprintterms}{variable}
    You have assigned an invalid value to \hyperlink{reserved:TAYLORPRINTTERMS}{\f{taylorprintterms}}.
    Allowed values are: an integer or the special identifier
    \f{all}.

\item[\msg{TAYLOR PACKAGE (...): this can't happen ...}]\mbox{}\\
    This message shows that an internal inconsistency was detected.
    This is not your fault, at least as long as you did not try to
    work with the internal data structures of \REDUCE. Send input
    and output to me, together with the version information that is
    printed out.

\end{description}

\subsection{Comparison to other packages}

At the moment there is only one \REDUCE{} package that I know of:
the extendible power series package by Alan Barnes and Julian Padget.
In my opinion there are two major differences:
\begin{itemize}
  \item The interface. They use the domain mechanism for their power
        series, I decided to invent a special kind of kernel. Both
        approaches have advantages and disadvantages: with domain
        modes, it is easier
        to do certain things automatically, e.g., conversions.
  \item The concept of an extendible series: their idea is to remember
        the original expression and to compute more coefficients when
        more of them are needed. My approach is to truncate at a
        certain order and forget how the unexpanded expression
        looked like.  I think that their method is more widely
        usable, whereas mine is more efficient when you know in
        advance exactly how many terms you need.
\end{itemize}



\section{TPS: extendible power series}

\index{Power series} \index{Extendible power series}\index{Power series!extendible}
\index{Barnes, Alan} \index{Padget, Julian}
\index{People!Barnes, Alan} \index{People!Padget, Julian}
\subsection{Introduction}
\index{Power series!expansions}
This package implements formal Laurent power series expansions in one
variable using the domain mechanism of REDUCE. This means that power
series objects can be added, multiplied, differentiated etc. like other
first class objects in the system. A lazy evaluation scheme is used in
the package and thus terms of the series are not evaluated until they
are required for printing or for use in calculating terms in other
power series. The series are extendible giving the user the impression
that the full infinite series is being manipulated.  The errors that
can sometimes occur using series that are truncated at some fixed depth
(for example when a term in the required series depends on terms of an
intermediate series beyond the truncation depth) are thus avoided.

The package was originally based on an earlier \emph{truncated power series} package
developed by Julian Padget in the 1980's. The name of the original package was
TPS and this was never changed. The alternative (more accurate) name EPS was
perhaps rejected because of possible confusion with the acronym for
\emph{encapsulated PostScript}.

In the first subsection below a brief description of the main operators
available for series expansion are given together with some examples of
their use.

\subsection{Basic Use}
\hypertarget{operator:PS}{}
\ttindextype{ps}{operator}
The most important operator is  \texttt{ps} which is used as follows:
\begin{verbatim}
  ps(EXP:algebraic, VAR:kernel, 
     ABOUT:algebraic):algebraic.
\end{verbatim}

The \texttt{ps} operator returns a Laurent power series object (a tagged domain
element) representing the univariate formal Laurent power series expansion of
\f{EXP} with respect to the dependent variable \f{VAR} about the expansion point
\f{ABOUT}.  \f{EXP} may itself contain power series objects.
If the function has a pole at the expansion point then the correct
Laurent series expansion will be produced.

The algebraic expression \f{ABOUT} should simplify to an expression
which is independent of the dependent variable \f{VAR}, otherwise
an error will result.  If \f{ABOUT} is the identifier \texttt{infinity}
then the power series expansion about $\infty$ is obtained in ascending
powers of \texttt{1/VAR}.

\textbf{Examples}
\begin{verbatim}
   a := ps(sin x, x, 0);
   ps(sin a, x, 0);
   ps(cos x/x^2, x, 0);
   ps(x/(1+x),x,infinity);
\end{verbatim}

\textbf{Operations on Power Series}
\index{Power Series!arithmetic}

As power series objects are domain elements they may be added, subtracted,
multiplied and divided in the normal way. For example if A and B are power
series objects with the \emph{same expansion variable and expansion point}:
\begin{verbatim}
    a+b; a*b;
    1/b; a/b;
\end{verbatim}

will produce power series objects representing the sum, product, reciprocal,
and quotient of the power series objects A and B
respectively.

\index{Power Series!differentiation}
\textbf{Differentiation}

Similarly, if A is a power series object depending on X then the input
{\tt df(a, x);} will produce the power series expansion of the
derivative of A with respect to X.

\index{Power series!integration}
\textbf{Integration}

The power series expansion of an integral may also be obtained (even if
REDUCE cannot evaluate the integral in closed form).  An example of
this is

\begin{verbatim}
    ps(int(exp(exp x),x),x,0);
\end{verbatim}

Note that if the integration variable is the same as the expansion
variable, the integration package is not called. If on the
other hand the two variables are different the integrator is
called to integrate each of the coefficients in the power series
expansion of the integrand.  The constant of integration is zero by
default.

Note that the Laurent series domain is not closed under integration with
respect to the expansion variable; if the term of degree -1 is non-zero a
logarithmic singularity error will occur on integration.

\index{Power series!exponentiation}
\textbf{Exponentiation}
The Laurent series domain is closed under exponentiation by an \emph{integer}
power. Thus, with respect to integer exponentiation, power series are first
class objects and for example the following results in automatic expansion of
the final result:
\begin{verbatim}
  a:= ps(cos x,x,0);
  b:= ps(sin x,x,0);
  a^2+b^(-2);
\end{verbatim}

However, for more general exponents automatic expansion does not occur. For
example given power series \texttt{a} and \texttt{b} defined as above, the
following commands are necessary:
\begin{verbatim}
    ps(a^(1/2),x,0);
    ps(a^pi,x,0);
    ps(a^b,x,0);
\end{verbatim}
Note \emph{any} power of a power series of \emph{order zero} (that is with a
non-zero term of degree zero) can be expanded as a power series (again of
order zero) provided only that the power is non-singular at the expansion point.
As the third example above shows the exponent may itself be a power series.

However in general the Laurent series domain is not closed under exponentiation.
If the result is to be a Laurent series some restrictions 
on the allowed values of the exponent and order of the original series are
necessary. Namely, if the order of the power series is non-zero ($\sigma$ say)
and the exponent is rational with denominator $q$ say, then $\sigma q$ must be
integral.  If the exponent is rational, but $\sigma q$ is not an integer,
a \emph{branch point} error is generated. For other exponents a
\emph{logarithmic singularity} error is usually generated. For example,
\begin{verbatim}
   a := ps(1-cos x,x,0);  % series has order 2
   ps(a^(1/2),x,0);   % series has order 1
   ps(a^(2/3),x,0);   % branch point error
   ps(a^pi,x,0);      % logarithmic singularity error
\end{verbatim}

\textbf{Power series of user defined functions}

New user-defined functions may be expanded provided the user provides
a rule or rule list defining the derivative of the function and optionally
its value at the expansion point. For example
\begin{verbatim}
    operator u;
    let df(u(~x),~x)= exp(e^x);
    let u(0) = e;
    ps(u(sin x),x,0);
\end{verbatim}
Of course the rules defined must be such that the function actually has a
Taylor series expansion about the specified point.

\textbf{Restrictions and Known Bugs}

Currently automatic expansion of quotients with an integer denominator does not
normally occur. One must use:
\begin{verbatim}
    a:=ps(sin x,x,0);
    ps(a/5,x,0);
or
    on rational;    % or on rounded;
    a/5;
\end{verbatim}

Currently the following does not produce a power series object (although the
result is formally valid):
\begin{verbatim}
   a := ps(cos x, x, 0);
   ps(2^a,x,0);
 % instead use:
   ps(2^cos x,x,0);
\end{verbatim}

If A is a power series object and X is a variable
which evaluates to itself then expressions such as \texttt{a*x} or
\texttt{int(a, x);}  do not automtically expand to a single power series object
(although the result returned is formally valid).  Instead expressions such as
 \texttt{ps(a*x,x,0)} and \texttt{ps(int(a,x),x,0} should be used.

Currently the handling of essential sigularities is rather erratic; sometimes
an Essential Singularity or Logarithmic Singularity error message is output,
but often the system fails rather ungracefully.

There is no simple way to write the results of power series
calculation to a file and read them back into REDUCE at a later
stage.

\textbf{Taylor Series Expansion}

The operator \texttt{pstaylor} may be used as follows:
\hypertarget{operator:PSTAYLOR}{}
\ttindextype{pstaylor}{operator}
\begin{verbatim}
  pstaylor(EXP:algebraic, VAR:kernel, 
           ABOUT:algebraic):algebraic.
\end{verbatim}

which uses the classic Taylor series algorithm for expanding \f{EXP} and
returning an extendible Taylor series object.

The \f{pstaylor} operator may be useful in contexts where the  operator \f{ps}
fails to build a suitable recurrence relation automatically and reports too
deep a recursion in \texttt{ps!:unknown!-crule}. A typical example is the
expansion of the $\Gamma$ function about an expansion point which is not a
non-positive integer
\footnote{Actually the TPS code now detects this case and automatically uses
\f{pstaylor} where appropriate.}.

Note, however, that \f{pstaylor} always returns a \emph{Taylor} series whose
order is non-negative.  Attempting to use \f{pstaylor} to expand a function
about a pole will fail with a zero divisor error message.

Also in many cases the use of an automatically generated recurrence relation
built by \f{ps} is more  efficient than using \f{pstaylor}, particularly if a
large number of terms is required; expansion of \f{tan} is a typical example
where the number of terms in the nth derivative grows exponentially.


\subsection{Printing Power Series}

If the command \f{ps} or \f{pstaylor} is terminated by a semi-colon, a power
series object is compiled and then a number of terms of the
power series expansion are evaluated and printed.

\textbf{psexplim Operator}

The expansion is carried out as far as the value specified by an internal
variable (with a default value of 6). This variable can be accessed via the
operator \texttt{psexplim}.
\hypertarget{operator:PSEXPLIM}{}
\ttindextype{psexplim}{operator}
\begin{verbatim}
  psexplim(UPTO:integer):integer.
or
  psexplim():integer
\end{verbatim}
If \texttt{psexplim} is called with an integer value, the internal variable is
updated to the value of \texttt{UPTO} and its previous value is returned.
If \texttt{psexplim} is called with no argument the current value is unaltered
and that value is returned. 

If \texttt{psexplim} is used to increase the expansion limit, sufficient
information is stored in the power series object to enable the additional
terms to be calculated without recalculating the terms already obtained.

If the command is terminated by a dollar symbol, a power series object
is compiled and the first term is calculated, but no output is printed.

\textbf{psprintorder Switch}
\hypertarget{switch:PSPRINTORDER}{}
\ttindexswitch[TPS]{psprintorder}

When the switch \sw{psprintorder} is ON the trailing terms of power series
beyond \texttt{psexplim} are represented in print by a big-O notation,
otherwise, three dots are printed. This switch is ON by default.
However, if expression being expanded is a polynomial in the expansion
variable and all non-zero terms have been output then the big-O or trailing
dots are omitted to indicate that the series is complete.

\subsection{Accessor Functions}
In this section a number of accessor functions which allow the user to extract
information such as the dependent variable, expansion point, a particular term
etc. of a power series object.

\textbf{psdepvar Operator}
\hypertarget{operator:PSDEPVAR}{}
\ttindextype{psdepvar}{operator}
\begin{verbatim}
  psdepvar(TPS:power series object):identifier.
\end{verbatim}
The operator \texttt{psdepvar} returns the expansion variable of the
power series object TPS. TPS should evaluate to a power
series object or an integer, otherwise an error results. If TPS
is an integer, the identifier \texttt{undefined} is returned.

\textbf{psexpansionpt operator}
\hypertarget{operator:PSEXPANSIONPT}{}
\ttindextype{psexpansionpt}{operator}
\begin{verbatim}
  psexpansionpt(TPS:power-series-object):algebraic.
\end{verbatim}
The operator \texttt{psexpansionpt} returns the expansion point of the
power series object TPS. TPS should evaluate to a power
series object or an integer, otherwise an error results. If TPS
is an integer, the identifier \texttt{undefined} is returned. If the
expansion is about infinity, the identifier \texttt{infinity} is
returned.

\textbf{psfunction Operator}
\hypertarget{operator:PSFUNCTION}{}
\ttindextype{psfunction}{operator}
\begin{verbatim}
  psfunction(TPS:power-series-object):algebraic.
\end{verbatim}
The operator \texttt{psfunction} returns the function whose expansion
gave rise to the power series object TPS. TPS should
evaluate to a power series object or an integer, otherwise an error
results.

\textbf{psterm Operator}
\hypertarget{operator:PSTERM}{}
\ttindextype{psterm}{operator}
\begin{verbatim}
  psterm(TPS:power-series-object, 
         NTH:integer):algebraic.
\end{verbatim}
The operator \texttt{psterm} returns the NTH term of the existing
power series object TPS. If NTH does not evaluate to
an integer or TPS to a power series object an error results.  It
should be noted that an integer is treated as a power series.

\textbf{psorder Operator}
\hypertarget{operator:PSORDER}{}
\ttindextype{psorder}{operator}
\begin{verbatim}
  psorder(TPS:power-series-object):integer.
\end{verbatim}
The operator \texttt{psorder} returns the order, that is the degree of
the first non-zero term, of the power series object TPS.
TPS should evaluate to a power series object or an error results. If
TPS is zero, the identifier \texttt{undefined} is returned.

\textbf{pstruncate Operator}
\hypertarget{operator:PSTRUNCATE}{}
\ttindextype{pstruncate}{operator}
\begin{verbatim}
  pstruncate(TPS:power-series-object,
             POWER:integer):algebraic.
\end{verbatim}
This procedure truncates the power series \texttt{TPS} discarding terms
of order higher than \texttt{POWER}. The series is extended automatically
if the value of \texttt{POWER} is greater than the order of last term
calculated to date. For example
\begin{verbatim}
    a := ps(sin x, x, 0);
    pstruncate(a, 11);
\end{verbatim}
will output the eleventh order polynomial resulting in truncating the series
for $sin x$ after the term involving $x^{11}$.

If \texttt{POWER} is less than the order of the series then $0$ is
returned.  If \texttt{POWER} does not simplify to an integer or if
\texttt{TPS} is not a power series object then a Reduce error result.


\subsection{Power Series Reversion}
\hypertarget{operator:PSREVERSE}{}
\ttindextype{psreverse}{operator}
\index{Power series!reversion}
In order to functionally invert a power series the operator \texttt{psreverse}
is used. 
\begin{verbatim}
    psreverse(TPS:power-series-object)
              :power-series-object
\end{verbatim}
Four cases arise:

\begin{enumerate}
\item If the order of the series is 1, then the expansion point of the
inverted series is 0.

\item If the order is 0 \emph{and} if the first order term in TPS
is non-zero, then the expansion point of the inverted series is taken
to be the coefficient of the zeroth order term in TPS.

\item If the order is -1 the expansion point of the inverted series
is the point at infinity.  In all other cases a REDUCE error is
reported because the series cannot be inverted as a power series.
Puiseux \index{Puiseux expansion} expansion would be required to
handle these cases.

\item If the expansion point of TPS is finite it becomes the
zeroth order term in the inverted series. For expansion about 0 or the
point at infinity the order of the inverted series is one.
\end{enumerate}

If TPS is not a power series object after evaluation an error results.

Some examples:
\begin{verbatim}
    ps(sin x,x,0);
    psreverse(ws); % produces series for asin x about x=0.
    ps(exp x,x,0);
    psreverse ws; % produces series for log x about x=1.
    ps(sin(1/x),x,infinity);
    psreverse(ws); % series for 1/asin(x) about x=0.
\end{verbatim}

\subsection{Power Series Composition}
\hypertarget{operator:PSCOMPOSE}{}
\ttindextype{pscompose}{operator}
\index{Power series!composition}
In order to functionally compose two power series the operator
\texttt{pscompose} is used. 
\begin{verbatim}
   pscompose(TPS1:power-series-object,
             TPS2:power-series-object)
             :power-series-object
\end{verbatim}
The power series TPS1 and TPS2 are functionally composed;
that is to say that TPS2 is substituted for the expansion
variable in TPS1 and the result expressed as a power series. The
dependent variable and expansion point of the result coincide with
those of TPS2.  The following conditions apply to power series
composition:

\begin{enumerate}
\item If the expansion point of TPS1 is 0 then the order of the
TPS2 must be at least 1.

\item If the expansion point of TPS1 is finite, it should
coincide with the coefficient of the zeroth order term in TPS2.
The order of TPS2 should also be non-negative in this case.

\item If the expansion point of TPS1 is the point at infinity
then the order of TPS2 must be less than or equal to -1.
\end{enumerate}

If these conditions do not hold the series cannot be composed (with
the current algorithm terms of the inverted series would involve
infinite sums) and a REDUCE error occurs.

Some examples:
\begin{verbatim}
  a:=ps(exp y,y,0);  b:=ps(sin x,x,0);
  pscompose(a,b);
  % Produces the power series expansion of exp(sin x)
  % about x=0.

  a:=ps(exp z,z,1); b:=ps(cos x,x,0);
  pscompose(a,b);
  % Produces the power series expansion of exp(cos x)
  % about x=0.

  a:=ps(cos(1/x),x,infinity);  b:=ps(1/sin x,x,0);
  pscompose(a,b);
  % Produces the power series expansion of cos(sin x)
  % about x=0.
\end{verbatim}

\subsection{pssum Operator}
\hypertarget{operator:PSSUM}{}
\ttindextype{pssum}{operator}
If an expression is known for the nth term of a power series, an extendible
power series object may be constructed by the operator \texttt{pssum}

\begin{verbatim}
   pssum(J:kernel = LOWLIM:integer, 
         COEFF:algebraic, X:kernel, 
         ABOUT:algebraic, POWER:algebraic)
         :power-series-object
\end{verbatim}

The formal power series sum for J from LOWLIM to \texttt{infinity} of
\begin{verbatim}
      COEFF*(X-ABOUT)**POWER
\end{verbatim}
when \texttt{ABOUT} is finite or zero, whereas if ABOUT is \texttt{infinity}
\begin{verbatim}
      COEFF*(1/X)**POWER
\end{verbatim}
is constructed and returned. This enables power series whose general
term is known to be constructed and manipulated using the other
procedures of the power series package.

J and X should be distinct simple kernels. The algebraics
ABOUT,  COEFF and POWER should not depend on the
expansion variable X, similarly the algebraic ABOUT should
not depend on the summation variable J.  The algebraic POWER should be
a strictly increasing integer-valued function of J for J in the range
LOWLIM to \texttt{infinity}.

Some examples:
\begin{verbatim}
   pssum(n=0,1,x,0,n*n);
   % Produces the power series summation for n=0 to
   % infinity of x**(n*n).

   pssum(n=1,n,x,0,n);
   % Produces the power series summation for n=1 to
   % infinity of n*x**n.

   pssum(m=1,(-1)**(m-1)/(2m-1),y,1,2m-1);
   % Produces a power series which is actually the expansion
   % of atan(y-1)  about y=1.

   pssum(j=1,-1/j,x,infinity,j);
   % Produces a power series which is actually the expansion
   % of log(1-1/x) about the point at infinity.

   pssum(n=0,1,x,0,2n**2+3n) + pssum(n=1,1,x,0,2n**2-3n);
   % Produces the power series summation for n=-infinity
   % to +infinity of x**(2n**2+3n).
\end{verbatim}
It should be noted that a formal power series is produced which may not have
a non-zero radius of convergence; the second example above illustrates this.
Nevertheless these formal series may be added, multiplied, differentiated etc.
by the TPS package. Of course, in general the result may also have a zero radius
of convergence.

\subsection{Miscellaneous Operators}

\textbf{pscopy Operator}
\hypertarget{operator:PSCOPY}{}
\ttindextype{pscopy}{operator}
\begin{verbatim}
   pscopy(TPS:power-series-object):power-series-object
\end{verbatim}
This procedure returns a copy of the power series \texttt{TPS}. 
The copy has no shared sub-structures in common with the original
 series.  This enables substitutions to be performed on the series
 without side-effects on previously computed objects. For example:
\begin{verbatim}
    clear a;
    b := ps(sin(a*x)), x, 0);
    b where a => 1;
\end{verbatim}

will result in \texttt{a} being set to 1 in each of the terms of the
power series and the resulting expressions being simplified. Owing to
the way power series objects are implemented using Lisp vectors, this
has the side-effect that the value of \texttt{b} is changed.  This may be
avoided by copying the series with \texttt{ pscopy} before applying the
substitution, thus:
\begin{verbatim}
    b := ps(sin(a*x)), x, 0);
    pscopy b where a => 1;
\end{verbatim}

\textbf{pschangevar Operator}
\hypertarget{operator:PSCHANGEVAR}{}
\ttindextype{pschangevar}{operator}
\begin{verbatim}
    pschangevar(TPS:power-series-object,
                X:kernel):power-series-object
\end{verbatim}
The operator \texttt{pschangevar} changes the dependent variable of the
power series object TPS to the variable X. TPS should evaluate to a power
series object and X to a kernel, otherwise an error results.
Also X should not appear as a parameter in TPS. The power series with the new
dependent variable is returned.

\textbf{psordlim Operator}
\hypertarget{operator:PSORDLIM}{}
\ttindextype{psordlim}{operator}
\begin{verbatim}
   psordlim(UPTO:integer):integer
or
   psordlim():integer
\end{verbatim}
An internal variable is set to the value of \texttt{UPTO} (which should
evaluate to an integer). The value returned is the previous value of
the variable.  The default value is 100.  If \texttt{psordlim} is called
with no argument, the current value is returned.

The significance of this control is that the system attempts to find
the order of the power series required, that is the order is the
degree of the first non-zero term in the power series.  If the order
is greater than the value of this variable an error message is given
and the computation aborts. This prevents infinite loops in certain cases,
for example:
\begin{verbatim}
    a:=ps(1-(cos x)^2,x,0);
    b :=ps((sin x)^2,x,0);
    b-a;
\end{verbatim}
This will also occur in the rather unlikely situation where the expression
being expanded is
\begin{enumerate}
\item identically zero, but is not recognized as such by REDUCE;
\item and its derivatives are not recognized as identically zero by Reduce;
\item  but the values of all derivatives at the expansion point are
simplified to zero by REDUCE.
\end{enumerate}





\newpage

\section{FPS: Automatic calculation of formal power series}
\indexpackage{FPS}

This package can expand a specific class of functions into their
corresponding Laurent-Puiseux series.\footnote{This package was written by Wolfram Koepf and Winfried Neun.}

\chapter[FPS: Formal power series]%
        {FPS: Automatic calculation of formal power series}
\label{FPS}
\typeout{[FPS: Formal power series]}

{\footnotesize
\begin{center}
Wolfram Koepf and Winfried Neun\\
Konrad--Zuse--Zentrum f\"ur Informationstechnik Berlin \\
Takustra\"se 7 \\
D--14195 Berlin--Dahlem, Germany \\[0.05in]
e--mail: Koepf@zib.de and Neun@zib.de
\end{center}
}

\ttindex{FPS}

This package can expand functions of certain type into
their corresponding Laurent-Puiseux series as a sum of terms of the form
\begin{displaymath}
\sum_{k=0}^{\infty} a_{k} (x-x_{0})^{k/n + s}
\end{displaymath}
where $s$ is the `shift number', $n$ is the `Puiseux number',
and $x_0$ is the `point of development'. The following types are
supported:
\begin{itemize}
\item
{\bf functions of `rational type'}, which are either rational or have a
rational derivative of some order;
\item
{\bf functions of `hypergeometric type'} where $a_{k+m}/a_k$ is a rational
function for some integer $m$, the `symmetry number';
\item
{\bf functions of `exp-like type'} which satisfy a linear homogeneous
differential equation with constant coefficients.
\end{itemize}

{\tt FPS(f,x,x0)}\ttindex{FPS} tries to find a formal power
series expansion for {\tt f} with respect to the variable {\tt x}
at the point of development {\tt x0}.
It also works for formal Laurent (negative exponents) and Puiseux series
(fractional exponents). If the third
argument is omitted, then {\tt x0:=0} is assumed.

Example: {\tt FPS(asin(x)\verb+^+2,x)} results in
\begin{verbatim}

         2*k  2*k             2  2
        x   *2   *factorial(k) *x
infsum(----------------------------,k,0,infinity)
        factorial(2*k + 1)*(k + 1)
\end{verbatim}
If possible, the output is given using factorials. In some cases, the
use of the Pochhammer symbol {\tt pochhammer(a,k)}$:=a(a+1)\cdots(a+k-1)$
is necessary.

{\tt SimpleDE(f,x)} tries to find a homogeneous linear differential
equation with polynomial coefficients for $f$ with respect to $x$.
Make sure that $y$ is not a used variable.
The setting {\tt factor df;} is recommended to receive a nicer output form.

Examples: {\tt SimpleDE(asin(x)\verb+^+2,x)} then results in
\begin{verbatim}
            2
df(y,x,3)*(x  - 1) + 3*df(y,x,2)*x + df(y,x)
\end{verbatim}

The depth for the search of a differential equation for {\tt f} is
controlled by the variable {\tt
fps\verb+_+search\verb+_+depth};\ttindex{fps\_search\_depth} higher
values for {\tt fps\verb+_+search\verb+_+depth} will increase the
chance to find the solution, but increases the complexity as well. The
default value for {\tt fps\verb+_+search\verb+_+depth} is 5.  For {\tt
FPS(sin(x\verb+^+(1/3)),x)}, or {\tt SimpleDE(sin(x\verb+^+(1/3)),x)}
{\em e.g.}, a setting {\tt fps\verb+_+search\verb+_+depth:=6} is necessary.

The output of the FPS package can be influenced by the\ttindex{TRACEFPS}
switch {\tt tracefps}.  Setting {\tt on tracefps} causes various
prints of intermediate results.



