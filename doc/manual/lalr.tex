
This package provides a parser-generator, somewhat styled after
\texttt{yacc} or the many programs available for use with other
languages. You present it with a phrase structure grammar and it
generates a set of tables that can then be used by the function
\texttt{yyparse} to read in material in the syntax that you specified.
Internally it uses a very well established technique known ``LALR'' which
takes the grammar are derives the description of a stack automaton that
can accept it. Details of the procedure can be found in standard books
on compiler construction, such as the one by Aho, Ullman, Lam and Sethi~\cite{dragoonBook}.

At the time of writing this explanation the code is not in its final form,
so this will describe the current state and include a few notes on what
might chaneg in the future.

Building a parser is done in Reduce symbolic mode, so say "\texttt{symbolic;}" 
or "\texttt{lisp;}" before starting your work.

\hypertarget{function:LALR_CREATE_PARSER}{}
To use the code here you use a function \texttt{lalr\_create\_parser}\ttindextype[LALR]{lalr\_create\_parser}{lisp function},
giving it two arguments. The first indicates precedence information and
will be described later: for now just pass the value \var{nil}.
The second argument is a list of productions, and the first one of these
is taken to be the top-level target for the whole grammar.

Each production is in the form
\begin{verbatim}
    (LHS   ((rhs1.1 rhs1.2 ...) a1.1 a1.2 ...)
           ((rhs2.1 rhs2.1 ...) a2.1 a2.2 ...)
           ...)
\end{verbatim}
\noindent which in regular publication style for grammars might be interpreted
as meaning
\begin{align*}
\text{LHS} & \Rightarrow & \text{rhs}_{1,1} \hspace{0.5em} \text{rhs}_{1,2} \ldots  & \left\{ \text{a}_{1,1} \hspace{0.5em} \text{a}_{1,2} \ldots \right\} \\
    & \hspace{0.5em}| & \text{rhs}_{2,1} \hspace{0.5em} \text{rhs}_{2,2} \ldots & \left\{ \text{a}_{2,1} \hspace{0.5em} \text{a}_{2,2} \ldots \right\} \\
    & \ldots  \\
    & \hspace{0.5em};
\end{align*}
The various lines specify different options for what the left hand side
(non-terminal symbol) might correspond to, while the items within the
braces are sematic actions that get obeyed or evaluated when the
production ruls is used.

Each LHS is treated as a non-terminal symbol and is specified as a simple
name. Note that by default the Reduce parser will be folding characters
within names to lower case and so it will be best to choose names for
non-terminals that are unambiguous even when case-folded, but I would like
to establish a convention that in source code they are written in capitals.

The RHS items may be either non-terminals (identified because they are
present in the left hand side of some production) or terminals. Terminal
symbols can be specified in two different ways.

The lexer has built-in recipes that decode certain sequences of characters
and return the special markers for !:symbol, !:number, !:string, !:list for
commonly used cases. In these cases the variable yylval gets left set
to associated data, so for instance in the case of !:symbol it gets set
to the particular symbol concerned.
The token type :list is used for Lisp or rlisp-like notation where the
input contains
    'expression
or  `expression
so for instance the input `(a b c) leads to the lexer returning !:list and
yylvel being set to (backquote (a b c)). This treatment is specialised for
handling rlisp-like syntax.

Other terminals are indicated by writing a string. That may either
consist of characters that would otherwise form a symbol (ie a letter
followed by letters, digits and underscores) or a sequence of
non-alphanumeric characters. In the latter case if a sequence of three or
more punctuation marks make up a terminal then all the shorter prefixes
of it will also be grouped to form single entities. So if "<-->" is a
terminal then '<', '<-' and '<--' will each by parsed as single tokens, and
any of them that are not used as terminals will be classified as !:symbol.

As well as terminals and non-terminals (which are writtent as symbols or
strings) it is possible to write one of

\begin{tabular}{l l}
    ({\ttfamily OPT} \textit{s1} \textit{s2} \ldots)      &     0 or 1 instances of the sequence \textit{s1}, \ldots \\
    ({\ttfamily STAR} \textit{s1} \textit{s2} \ldots)     &     0, 1, 2, \ldots instances. \\
    ({\ttfamily PLUS} \textit{s1} \textit{s2} \ldots)     &     1, 2, 3, \ldots instances. \\
    ({\ttfamily LIST} \textit{sep} \textit{s1} \textit{s2} \ldots) &     like ({\ttfamily STAR} \textit{s1} \textit{s2} \ldots) but with the single item \\
                            &     \textit{sep} between each instance. \\
    ({\ttfamily LISTPLUS} \textit{sep} \textit{s1} \ldots) &    like ({\ttfamily PLUS} \textit{s2} \ldots) but with \textit{sep} interleaved. \\
    ({\ttfamily OR} \textit{s1} \textit{s2} \ldots)        &    one or other of the tokens shown.
\end{tabular}

When the lexer processes input it will return a numeric code that identifies
the type of the item seen, so in a production one might write
    (!:symbol ":=" EXPRESSION)
and as it recognises the first two tokens the lexer will return a numeric
code for !:symbol (and set yylval to the actual symbol as seen) and then
a numeric code that it allocates for ":=". In the latter case it will
also set yylval to the symbol !:!= in case that is useful.
%
Precedence can be set using \texttt{lalr\_precedence}. See examples below.



\subsection{Limitations}
\begin{enumerate}
\item Grammar rules and semantic actions are specified in fairly raw Lisp.
\item The lexer is hand-written and can not readily be reconfigured for
    use with languages other than rlisp. For instance it has use of "!"
    as a character escape built into it.
\end{enumerate}


\subsection{An example}
\begin{verbatim}
% Here I set up a sample grammar
%    S' -> S
%    S  -> C C        { }
%    C  -> "c" C      { }
%        | "d"        { }
% This is example 4.42 from Aho, Sethi and Ullman's Red Dragon book.
% It is example 4.54 in the more recent Purple book.
%
%
grammar := '(
  (s  ((cc cc)  )   % Use default semantic action here
  )
  (cc (("c" cc) (list 'c !$2))   % First production for C
      (("d")    'd           )   % Second production for C
  ))$

parsertables := lalr_create_parser(nil, grammar)$

<< lex_init();
   yyparse() >>;
c c c d c d ;
\end{verbatim}

\endinput


A grammar is given as a list of rules. Each rule has a single non-terminal
and then a seqence of productions for it. Each production can optionally
be provided with an associated semantic action.

\begin{verbatim}
 GRAMMAR ::= "(" GTAIL
         ;

 GTAIL   ::= ")"         % Sequence of rules
         |   RULE GTAIL
         ;

 RULE    ::= "(" nonterminal RULETAIL
         ;

 RULETAIL::= ")"         % Sequence of productions
         | PRODUCTION RULETAIL
         ;

 PRODUCTION ::= "(" "(" PT1 PT2
         ;

 PT1     ::= ")"         % Symbols to make one production
         | nonterminal PT1
         | terminal PT1
         ;

 PT2     ::= ")"         % Semantic actions as Lisp code
         | lisp-s-expression PT2
         ; 
\end{verbatim}

\hypertarget{function:LALR_PRECEDENCE}{}
\ttindextype[LALR]{lalr\_precedence}{lisp function}
Before specifying a grammar it is possible to declare the precedence and
associativity of some of the terminal symbols it will use. Here is an
example:
\begin{verbatim}
 lalr_precedence '("." !:right "^" !:left ("*" "/") ("+" "-"));
\end{verbatim}
\noindent will arrange that the token ``\verb@.@'' has highest precedence
and that it is left associative. Next comes ``\verb@^@'' which is right
associative.
Both ``\verb@*@'' and ``\verb@/@'' have the same precedence and both are
left associative, and finally ``\verb@+@'' and ``\verb@-@'' are also equal
in precedence but lower than ``\verb@*@''.

\hypertarget{function:LALR_CONSTRUCT_PARSER}{}
\ttindextype[LALR]{lalr\_construct\_parser}{lisp function}
With those precedences in place one could specify a grammar by
\begin{verbatim}
 lalr_construct_parser '(
    (S  ((!:symbol))
        ((!:number))
        ((S "." S))
        ((S "^" S))
        ((S "*" S))
        ((S "/" S))
        ((S "+" S))
        ((S "-" S))));
\end{verbatim}

The special markers \verb+!:symbol+ and \verb+!:number+ will match any
symbols or numbers -- for commentary on what count as such see the discussion
of the lexter later on. The strings stand for fixed tokens and by virtue
of them being used in the grammer the lexer will recognise them specially.

