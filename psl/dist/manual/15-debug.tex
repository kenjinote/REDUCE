\chapter{Debugging Tools}

\section{The Debug Module}

\index{trace}\index{break}
 {\bf debug} is a loadable  option.   This module provides a
basic debugging tool which provides a breakpoint  function,  and
functions   for  breaking  and  tracing  functions  and  methods
selected by the user.  debug is loaded automatically if one of the
functions tr, trst or br is called.

\subsection{Overview of Functionality}

  The function breakpoint allows the user to  set  a  breakpoint
anywhere  within  his  code.   Whenever a breakpoint is called a
continuable break loop is entered.   This  allows  the  user  to
examine  or  modify  variables,  view  a backtrace, and continue
execution.

  Tracing gives a record of the sequence of calls made to a  set
of  functions  specified by the user.  When a traced function is
entered a message is printed giving the function name, level  of
recursion,  and  values  of  any  arguments.  Upon function exit
another message is printed, giving the  value  returned  by  the
function.

  Breaking a function means that the system enters a continuable
break  loop  both before and after evaluation of the body of the
function.  Within the break loop the user may inspect and change
values of parameters and non-local variables, and then  continue
execution.   A break on entry occurs after the function's formal
parameters have been bound to the actual  arguments  but  before
any  of the forms in the function body have been evaluated.  The
break on exit occurs after the body of the broken  function  has
been evaluated.

\subsection{Using Break and Trace}


\de{tr}{(tr [NAME: id]): list}{macro}
{    If  no  arguments  are  given  then a list is returned which
    indicates  which  functions  are  currently  being   traced.
    Otherwise,  each  argument  is  treated  as a reference to a
    function, and if  possible,  the  function  is  set  up  for
    tracing.    The   list  which  is  returned  represents  the
    functions which have been added  to  the  set  of  functions
    being  traced.   
}

\de{trst}{(trst [NAME: id]): list}{macro}
{    If  no  arguments  are  given  then a list is returned which
    indicates  which  functions  are  currently  being   traced.
    Otherwise,  each  argument  is  treated  as a reference to a
    function, and if  possible,  the  function  is  set  up  for
    tracing.    The   list  which  is  returned  represents  the
    functions which have been added  to  the  set  of  functions
    being  traced. Trst can be used only for functions which are
    not compiled. Additionally to the entry and exit messages,
    for each $setq$ assignment inside the function the variable name
    and the value are printed.
}

\de{br}{(br [NAME: id, method-specification]): list}{macro}
{    If  no  arguments  are  given  then a list is returned which
    indicates which functions are currently broken.     NAME is
    treated as a reference to a function, and if  possible,  the
    function  is set up for breaking.  Subsequent arguments will
    be processed.  The list which  is  returned  represents  the
    functions  which  have  been  added  to  the  set  of broken
    functions.  
}

\de{untr}{(untr [NAME:id]): list}{macro}
{    If no arguments are given then tracing is removed  for  each
    traced  function.   NAME  is  treated  as  a  reference to a
    function, and if possible,  tracing  is  disabled  for  this
    function.  Subsequent arguments will be processed.  The list
    which  is  returned  contains the names of the functions for
    which tracing has been removed.    
}

\de{unbr}{(unbr [NAME:id,]): list}{macro}
{    If no arguments are given then breaking is  turned  off  for
    each  broken  function.  NAME is treated as a reference to a
    function, and if possible, breaking  is  disabled  for  this
    function.   The list which is returned contains the names of
    the functions for which breaking  has  been  removed.    
}


\subsection{Sample Session}


  Trace  and break printout goes to the standard output channel.
Because multiple nested calls are usually shown, indentation and
vertical bars are used to line  up  function  entry  and  return
messages for the same function.

\begin{verbatim}
(de add3 (a1 a2 a3)
  (+ a1 (add2 a2 a3)))
\end{verbatim}
\begin{verbatim}
(de add2 (a1 a2)
  (+ a1 a2))
\end{verbatim}
\begin{verbatim}
(de fact (n)
  (cond ((zerop n) 1)
        (t (times n (fact (sub1 n))))))
\end{verbatim}
\begin{verbatim}
1 lisp> (br add3)
In a break, t prints backTrace; c Continues; 
q Quits a level; a Aborts to top.
(ADD3)
2 lisp> (tr 'add3)
***** (QUOTE ADD3) is not a valid method-specification.
NIL
3 lisp> (tr add3 add2)
(ADD2 ADD3)
4 lisp> (add3 5 6 7)
ADD3 entry:
|     A1: 5
|     A2: 6
|     A3: 7
|  ADD3 entry BREAK:
|  NMODE Break loop
5 lisp break (1)> c     % Continue
|  |  ADD2 entry:
|  |  |     A1: 6
|  |  |     A2: 7
|  |  ADD2 value = 13
|  ADD3 exit BREAK, value = 18:
|  NMODE Break loop
6 lisp break (1)> c
ADD4 value = 18
18
\end{verbatim}
\begin{verbatim}
1 lisp> (tr fact)
(FACT)
2 lisp> (fact 4)
FACT entry:
|     N: 4
|  FACT reentry (# 2):     
|  |        % (# 2) represents the depth of recursion
|  |     N: 3
|  |  FACT reentry (# 3):
|  |  |     N: 2
|  |  |  FACT reentry (# 4):
|  |  |  |     N: 1
|  |  |  |  FACT reentry (# 5):
|  |  |  |  |     N: 0
|  |  |  |  FACT value (# 5) = 1
|  |  |  FACT value (# 4) = 1
|  |  FACT value (# 3) = 2
|  FACT value (# 2) = 6
FACT value = 24
24
3 lisp> (trace)
(FACT ADD2 ADD3 ADD4)
4 lisp> (untrace)
(ADD4 ADD3 ADD2 FACT)
5 lisp> (restr)
T
\end{verbatim}


  Tracing  a  macro  will happen at macroexpand time.  The value
returned will be the  macroexpanded  form,  and  thus  is  quite
useful  in  determining  if  your macro expanded properly.  Note
that when code  is  complied  the  application  of  a  macro  is
expanded and this expansion replaces the the application.

\begin{verbatim}
1 lisp> (trace plus)
(PLUS)
2 lisp> (plus 2 9 -3 7 8)
PLUS entry:
|     ARG1 :(PLUS 2 9 -3 7 ...)
PLUS value = (PLUS2 2 (PLUS2 9 (PLUS2 -3 (PLUS2 7 8))))
23
\end{verbatim}
\begin{verbatim}
(de fact (n)
  (cond ((zerop n) 1)
        (t (breakpoint "Fact; n<>0 branch of cond, n=%p" n)
           (times n (fact (sub1 n))))))
\end{verbatim}
\begin{verbatim}
1 lisp> (fact 2)
User Breakpoint: Fact; n<>0 branch of cond, n=2
NMODE Break loop
2 lisp break (1)> c
User Breakpoint: Fact; n<>0 branch of cond, n=1
NMODE Break loop
3 lisp break (1)> c
2
\end{verbatim}

\subsection{Redefining a Broken or Traced Function}

  The  basic  definition of a function is the definition without
any of the break or tracing side effects.  This basic definition
may be accessed by the function getd.  The basic definition  may
also be redefined by using the function putd without interfering
with the break/trace side effects, as long as the parameter list
stays  the same.  If you intend to change the number or order of
parameters you should first remove the break/trace wrappers from
it with untrace and/or unbr.


